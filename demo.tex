\documentclass[10pt]{beamer}

\usetheme[progressbar=frametitle]{metropolis}
\usepackage{appendixnumberbeamer}

\usepackage{booktabs}
\usepackage[scale=2]{ccicons}

\usepackage{pgfplots}
\usepgfplotslibrary{dateplot}

\usepackage{xspace}
\newcommand{\themename}{\textbf{\textsc{metropolis}}\xspace}

\title{Copilots}
\subtitle{Tolerating Slowdowns in Replicated State
Machines}
% \date{\today}
\date{}
\author{Khiem Ngo, Siddhartha Sen, Wyatt Lloyd}
\institute{Princeton University, Microsoft Research - Presented at OSDI '20}
\titlegraphic{\hfill\includegraphics[height=1.5cm]{logo.png}}

\begin{document}

\maketitle

\begin{frame}{Table of contents}
  \setbeamertemplate{section in toc}[sections numbered]
  \tableofcontents%[hideallsubsections]
\end{frame}

\section[Introduction]{Introduction}

\begin{frame}[fragile]{Overview}
\textbf{Copilots} is a State Machine Replication (SMR) protocol aimed at tolerating replicas slowdowns. \\
\smallbreak
It is the first 1-slowdown-tolerant consensus protocol implementing SMR.
\end{frame}


\begin{frame}[fragile]{State-of-the-art}
2 types of majority-based replication protocols: 
\begin{itemize}
    \item \textbf{single leader:} Multi-Paxos
    \item \textbf{multiple leaders:} EPaxos
\end{itemize}
Those 2 protocols implement State Machine Replication that allows for $n=2f+1$ up to $f$ faults. \\
\smallbreak
However, if a replica $p \in n$ is not faulty but \textbf{slow} (misconfiguration, network issues, hardware failures...), it results in a \textbf{high latency}.
\end{frame}


\begin{frame}[fragile]{What's Copilots?}
 \textbf{Copilots} 1-slowdown-tolerant replication protocol with: 
 \begin{itemize}
     \item  \textbf{Competitive performance} in case of no slowdown
     \begin{itemize}
         \item low latency
         \item high throughput
     \end{itemize}
     \item \textbf{SMR properties}
     \begin{itemize}
         \item fault tolerance
         \item linearizability
     \end{itemize}
     \item \textbf{High performance} with 1-replica slowdown compared to state-of-the-art algorithms (Multi-Paxos and EPaxos)
 \end{itemize}

\end{frame}

\begin{frame}[fragile]{What's a slowdown?}
A replica $p$ is considered \textbf{slow} when its response to messages takes more than a threshold time $t$ over its normal response time.\\
\smallbreak
Only the time between when the machine receives a request  and sends a response back is considered.
\smallbreak
\textbf{1-slowdown-tolerance:} a  SMR protocol is 1-slowdown-tolerance if the response time T to a client request with 1 slow replica is close to the response time T' without any slow replica.
\end{frame}

\begin{frame}[fragile]{Existing protocols slowdown}
  \begin{columns}[T,onlytextwidth]
    \column{0.5\textwidth}
    \begin{center}
        \textbf{Multi-Paxos}
    \end{center}
    \column{0.5\textwidth}
    \begin{center}
        \textbf{EPaxos}
    \end{center}
  \end{columns}

\end{frame}


{\setbeamercolor{palette primary}{fg=black, bg=yellow}
\begin{frame}[standout]
  Questions?
\end{frame}
}
\end{document}
